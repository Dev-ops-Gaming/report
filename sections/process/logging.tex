Logging in MiniTwit was implemented using Dozzle. Dozzle is a lightweight web-based log viewer for Docker containers that also easily supports a swarm setup. When we initially wanted to implement logging, we implemented an ELK stack in our standard Docker Compose. This was a very basic implementation with no parsing or user setup. However, this never made it to production because while it was being worked on, we implemented Docker Swarm, and suddenly, a lot changed for our system. Meanwhile, another group member noticed a TA proposing a look at Dozzle if you had problems with ELK. Dozzle was implemented with great success in no time and worked great for us.

\subsubsection*{What do we log in Minitwit?}
In Minitwit, we log application events using Go's standard logging functions, including error messages from our main application and API services. Our database layer (GORM) is configured to log at a warning level. We also log API error responses and critical application failures.

\subsubsection*{How do we aggregate logs?}
All logs in Minitwit are aggregated at the container level using Docker’s built-in logging, with every service writing to \texttt{stdout} and \texttt{stderr}. We use Dozzle, deployed in global mode in our swarm, to view and search logs from all running services in real time through its web interface. 

\subsubsection*{Why we choose Dozzle}
We ended up using Dozzle because it enabled us to filter logs by container and log level. We have the capacity to transmit logs from all of our services in real time and filter them by service and different logging levels. A sidebar is available, which is equipped with a fuzzy search ability and provides a comprehensive overview of each service, node, replica, and swarm manager. It offers a standard search within the logs for manual searches, as well as the ability to conduct Regex searches within the logs. Dozzle is a simple and effective method of logging, as it is not infrastructure-intensive and is extremely lightweight. Implementation within a Docker swarm is effortless with Dozzle. Dozzle was the optimal choice for us due to its user-friendly interface and all of the aforementioned.

\subsubsection*{ELK stack features we missed out on}
We are unable to implement a role-based access control or authentication system with Dozzle, which is not suitable for production. As of now, unauthorized users can see our logs, and that is, as mentioned, potentially a security breach. We cannot save our logs indefinitely, and cannot distinguish between what should be parsed and what should not. Additionally, we are forfeiting the performance that Elasticsearch would have offered through its robust indexing and search capabilities.
\\\\
While there is a lot more we miss out on not using an ELK stack, these are the ones we feel were relevant to us. An ELK stack would require a lot more management, and you could almost endlessly improve it. We feel that Dozzle provided us with what we needed. Combined with our monitoring, we have the opportunity as developers to quickly gain an overview of bottlenecks or bugs in the application and fix them accordingly. 