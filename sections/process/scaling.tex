\subsubsection{Scaling}
To scale our Minitwit application, we performed horizontal scaling using Docker Swarm. We created three manager nodes and four worker nodes. \\
We have replicas of the following services:
\begin{center}
\begin{tabular}{ |c|c|c| } 
 \hline
 Service & Replicated/Global & No. of replicas \\ 
 \hline
 Minitwit App & Replicated & 3 \\ 
 \hline
 Minitwit API & Replicated & 4 \\
 \hline
 Prometheus & Global & N/A \\ 
 \hline
 Dozzle & Global & N/A \\ 
 \hline
\end{tabular}
\end{center}

Prometheus and Dozzle are set to be global services, meaning one instance of each runs on every node in the swarm. This ensures that monitoring and logging are consistently performed on all nodes, preventing any loss of critical information, whether it is logging or monitoring, due to missing coverage. 

\subsubsection{Upgrades}
We use a rolling update strategy with a start-first order to update the services in our swarm. This means one replica at a time is updated by starting a new container before stopping the old one, ensuring minimal downtime. Updates are monitored for 30 seconds, and if any failures are detected, our swarm will automatically roll back to the previous working version \cite{dockerComposeDeploy}.