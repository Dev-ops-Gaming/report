\subsubsection{Risk identification}
Our assets:
\begin{itemize}
    \item Web app and API
    \item PostgreSQL database (Hosted through DigitalOcean)
    \item Monitoring (Prometheus and Grafana)
    \item Logging (Dozzle)
    \item DigitalOcean VMs
\end{itemize}

Threat Sources:
\begin{itemize}
    \item SQL Injection
    \item Exposed ports 
    \item Vulnerable password hashing
    \item No requirements for user password (length, symbols, etc.)
    \item Vulnerable dependencies
    \item No password on Dozzle (logging)
    \item Cross-Site Scripting (XSS)
\end{itemize}

Risk Scenarios:
\begin{enumerate}[label=\textbf{\Alph*}]
    \item Attacker performs SQL injection to download sensitive user data
    \item Attacker performs SQL injection to delete information from the database
    \item Attacker performs SQL injection to log in to a legitimate user's account
    \item Attacker brute-forces the hash of a password, letting them log in to a legitimate user's account
    %\item Attacker get access to a developers credentials
    %\item Attacker uses the open ports to overload our Prometheus with queries
    \item Attacker connects externally to our database, either stealing, modifying or deleting data
    \item Attacker intercepts messages sent over the network, as they are not encrypted
    \item Attacker exploits a vulnerable dependency, which lets them perform some malicious attack. As the actual threat level depends on the specific vulnerability, we've chosen to be pessimistic and assume that the consequences will be disastrous
    \item Attacker accesses our logs on Dozzle, which lets them see all logs
    \item Attacker uses Cross-Site Scripting to run malicious code
    \item Attacker performs a DDoS attack
\end{enumerate}

\subsubsection{Risk Analysis}
Here are the scenarios presented on the risk matrix:
\begin{center}
\begin{tabular}{ |c|c|c|c|c| } 
 \hline
  & Rare & Unlikely & Likely & Certain \\ 
 \hline
 Catastrophic & B &  &  & G\\ 
 \hline
 Critical & A &  &  & \\ 
 \hline
 Marginal & &  &  F& E, I\\ 
 \hline
 Negligible & C &  & D, J & H\\ 
 \hline
\end{tabular}
\end{center}
\subsubsection{Discussion of risks}
Scenarios A, B, and C have a 'Rare' probability, as we already utilize the ORM-library 'GORM'. GORM uses prepared statements and automatically escapes arguments to avoid SQL injection. Without GORM, the probability would be much higher. Specifically for scenario B, it would also have been a good idea to have a backup of the database. The severity is also not a high risk, as the passwords are encrypted and no other sensitive user data is stored.\\
Scenario F was dealt with using the TLS protocol. Our Minitwit application using TLS is hosted at https://lukv.dk.\\
For scenario G, we have added Dependabot, which will automatically search for outdated dependencies, and create pull requests to update such dependencies.
\\\\
For Scenario E, we should have set up some sort of firewall (which we could do on DigitalOcean), such that only specific services are allowed to establish a connection to the database. 
%For scenario E, we would have to perform a workaround, as the exposed ports are caused by Docker and UFW being incompatible, an issue which has not yet been patched. However, Prometheus is the only service that is exposed, and the only threat is Prometheus crashing due to an overload of queries. Therefore, it is not considered a high-priority threat.\\
Scenario I can never truly be prevented, but there are ways to mitigate the risk, such as sanitizing inputs. 
\\
Scenario H can be prevented by locking our Dozzle interface with a password. However, no sensitive data is logged in Dozzle, and thus it is considered low priority.