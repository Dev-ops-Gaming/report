We are using GitHub Actions to automate the testing and deployment of Minitwit. There are 5 GitHub Actions workflows:
\begin{enumerate}
    \item Deploy services to DigitalOcean
    \item Use Hadolint on dockerfiles
    \item Run golangci-lint tool
    \item Run static tools for staging 
    \item Release MiniTwit (automatically)
\end{enumerate}

\subsubsection{Use Hadolint on dockerfiles}
This workflow only runs when it get called by other workflows, and analyse if the dockerfiles \code{docker/Dockerfile} and \code{docker/Dockerfile.api} contains any linting issues.\\
The steps for this workflow is:
\begin{outline}[enumerate]
    \1 \textbf{Checkout}
        \2 \textbf{Purpose} 
            \3 The workflow uses \code{actions/checkout@v2} to fetch the code from the repository.
        \2 \textbf{Steps}
            \3 Checkouts the code.

        
    \1 \textbf{Hadolint Action app}
        \2 \textbf{Purpose}
            \3 Checks for any linting errors using \code{hadolint/hadolint-action@v3.1.0} in \code{docker/Dockerfile}.
        \2 \textbf{Steps}
            \3 Runs \code{hadolint-action} on \code{docker/Dockerfile}.
    
        
    \1 \textbf{Hadolint Action api}
        \2 \textbf{Purpose}
            \3 Checks for any linting errors using \code{hadolint/hadolint-action@v3.1.0} in \code{docker/Dockerfile.api}.
        \2 \textbf{Steps}
            \3 Runs \code{hadolint-action} on \code{docker/Dockerfile.api}.
\end{outline}


\subsubsection{Run golangci-lint tool}
This workflow also only runs when it get called by other workflows, and analyses the go source code any linting issues.
\begin{comment}
\begin{outline}[enumerate]
    \1 \textbf{Environment setup}
        \2 defines the \code{GO_VERSION} to be stable and \code{GOLANGCI_LINT_VERSION}
    \1 test
\end{outline}
\end{comment}
\begin{outline}[enumerate]
    \1 \textbf{Environment setup}
        \2 \textbf{Purpose}
            \3 Defines some environment variables that are used in the workflow.
        \2 \textbf{Steps}
            \3 \code{GO\_VERSION = stable}
            \3 \code{GOLANGCI\_LINT\_VERSION = v1.64}

            
    \1 \textbf{Detect Modules}
        \2 \textbf{Purpose}
            \3 To output all the Go modules in \code{./minitwit}.
        \2 \textbf{Steps}
            \3 Checkout the repository.
            \3 Runs \code{go list -m} to list all Go modules and outputs in JSON format. 
    
    \1 \textbf{Format Go Files}
        \2 \textbf{Purpose}
            \3 To format all the \code{.go} files, done by using \code{Jerome1337/gofmt-action@v1.0.5}.
        \2 \textbf{Steps}
            \3 Checkout the repository.
            \3 Runs \code{gofmt-action} to verify formatting in \code{./minitwit}.

            
    \1 \textbf{Golangci Lint}
        \2 \textbf{Dependency}
            \3 Needs \textit{Detect Modules} to run.
        \2 \textbf{Steps}
            \3 Checkouts the repository and setups a go environment.
            \3 Runs \code{golangci-lint} in each GO module.
\end{outline}
\subsubsection{Deploy services to DigitalOcean}
This workflow runs every time the main branch gets a push.
\begin{outline}[enumerate]
    \1 \textbf{call-hadolint and call-golangci}
        \2 \textbf{Dependency}
            \3 \textit{Hadolint on dockerfiles} and \textit{golangci-lint}.
        \2 \textbf{Purpose} 
            \3 to ensure code quality before deployment.
        \2 \textbf{Steps}
            \3 Checkout the repository.
            \3 Run \code{Hadolint on dockerfiles}.
            \3 Run \code{golangci-lint}.
    \1 \textbf{Run tests}
        \2 \textbf{Dependency}
            \3 \code{call-hadolint and call-golangci}.
        \2 \textbf{Purpose}
            \3 To test the code to ensure code quality before deployment.
        \2 \textbf{Steps}
            \3 Checkout the repository.
            \3 Setup a Go environment with v1.22 of Go.
            \3 Setup docker Compose.
            \3 Find the test script (\code{run\_tests.sh}).
            \3 Run the test script.
    \1 \textbf{Build And Deploy}
        \2 \textbf{Dependency}
            \3 \textit{Run tests}
        \2 \textbf{Purpose}
            \3 Build a Docker image, push it to Docker Hub and Deploy it on Digital Ocean Droplet.
        \2 \textbf{Steps}
            \3 Checkout the repository.
            \3 Verify secrets.
            \3 Login to Docker Hub.
            \3 Set up Docker Buildx.
            \3 Build and push minitwit-app.
            \3 Build and push minitwit-api.
            \3 Configure SSH.
            \3 Sync files with rsync.
            \3 Deploy to server.
\end{outline}

\subsubsection{Release MiniTwit (automatically)}
This workflow creates a release on Github every thursday at 23:30 UTC.
\begin{outline}
    \1 \textbf{Automatic Release}
        \2 \textbf{Purpose} 
            \3 Creates a release on Github.
        \2 \textbf{Steps}
            \3 Checkout the repository.
            \3 Fetch latest version tag.
            \3 Determine next version.
            \3 Validate against SemVar pattern.
            \3 Generate release notes.
            \3 Create GitHub Release.
\end{outline}